\documentclass{book}
\usepackage[a4paper,left=3cm,right=3cm,top=3cm,bottom=3cm]{geometry}
\usepackage{amsmath}
\usepackage{amssymb}
\usepackage{hyperref}

\usepackage{adjustbox}
\usepackage{wrapfig}
\usepackage{tkz-euclide}
\usepackage[english, russian]{babel}
\usepackage{array}
\usepackage{multicol}

\usepackage{etoolbox}
\makeatletter
\patchcmd{\chapter}{\if@openright\cleardoublepage\else\clearpage\fi}{}{}{}
\makeatother

\usepackage{fontspec}
\setmainfont{Linux Libertine O}
\usepackage{unicode-math}
\setmathfont{Cambria Math}

\title{
\textit{\small{Georgii Potoshin, 2024}}\\
\vspace{0.3ex}
\textit{\huge{Высшая Алгебра, Николай Вавилов}}\vspace{1ex}
}

\date{\vspace{-10ex}}

\begin{document}
\maketitle
\section{Предыстория}
Определить, что такое современная алгебра трудно. Всё, что рассказывалось в
школе – вещи очень древние ~1000 лет. Один польский математик в своей книге
"Математический калейдоскоп" описывал распространение математических знаний
следующим образом. Человечество по отношению к математическим знаниям образует
очень растянутую процессию. Математические знания, которыми располагает
большинство людей предшествует эпохи строительства египетских пирамид. Что по
сути изучают в школе (по алгебре, анализ конечно ушел чуть дальше, к
экспонентам и логарифмам).
\begin{itemize}
    \item \[ax+b=0\]
    \item \[ax^2+bx+c=0\]
    \item \[
        \begin{cases}
            ax + by = 0\\
            cx + by = 0
        \end{cases}\] 
\end{itemize}
Есть египетские тексты, что давали методы решения этих проблем. Так же есть
китайский тексты 5 века до нашей эры, что описывают решение произвольных
линейных систем методом, что сегодня носит имя Гаусса (за 22 века до него).
В этом направлении алгебра долго развивалась в античные века и апофеозом стала
книга Диофанта "Арифметика", правда там решалась иная задача, а именно решение
уравнений в целых числах.

В 3-4 веках греческие ученые побежали в Индию, Иран (от христианства). Именно
поэтому большинство математических текстов этого периода написано на арабском.
Так книга, в честь которой названа это дисциплина, хоть и написана на арабском,
но её автор – Арихизмий был персом.

Переломным моментом для европейской науки (в частности по мнению Феймана, его
тогда спросили об этом на одном из интервью и он назвал изобретение комплексных
чисел) – конец 15 века, стал первый прогресс, была решена задача, непосильная
грекам.
\[ ax^3 + bx^2 + cx + d = 0\;\text(Амархаям) \]
Была решена Итальянским алгебраистом, а сама история совершенно криминальная.

Мы говорим, что математика – это наука, но это сильное преувеличение.
Математика – это совершенно иной вид деятельности, гораздо больше свободы, чем
в науке.

Так вот, оказалось, что при решении кубических уравнений с действительными
коэффициентами необходимо проводить вычисления в комплексных числах. Это был
переломный момент европейской цивилизации. До этого считалось, что всё знали
древние, но $\mathbb{C}$ – реальный прорыв. Любая современная электроника,
правда как стало понятно в 19 веке, основана на $\mathbb{C}$. Затем математика
начала набирать обороты. Виет ввёл буквы, Де Ферма ввел двойные числа, другое
расширение $\mathbb{R}\cup\{d|d^2 = 0\}$, алгебризация актуальных бесконечно
малых (17 век) [такое даже в стандартных курсах анализа не изучают]. Фон Лейбниц
придумал детерминант (17 век). В 18 веке решались любые линейные системы, но
уравнения более высоких степеней – нет. В 18 веке аналитически доказали основную
теорему Алгебры. Началась алгебраическая теория чисел с работ Лежандра и Гаусса.
В области решения уравнений Абелем и Руфини было доказано, что общее уравнение
не решается в радикалах, то есть нет общей формулы. Затем Галуа придумал
теорию Галуа, конечные поля, простые группы и эллиптические интегралы. Он дал
ответ на вопрос, когда уравнения любой степени решаются в радикалах. Он ввел
понятие группы и поля и с этого момента понимание алгебры стало меняться. В
целом теория Галуа решает многие классические задачи, как трисекция угла
например, так что является общекультурной вещью и странно, что она не вошла в
стандартные курсы, хотя есть один учебник, которой предлагает рассказывать
теорию Галуа школьникам. "Я не верю, что образование развивается. Если вы
зайдете в лабораторию алгебры и теории чисел в Париже, там вы увидите учебник
для гимназий, написанный Д.А.Граве в 5 году. В этом учебнике излагалась теория
Галуа (1830 г.) Как образование оторвано от науки!".

После теории Галуа содержание алгебры стало немного смещаться. На самом деле
любое алгебраическое уравнение может быть решено, не в радикалах разумеется.
Формула для 5 степени была известна Гамельтону и Изенштейну. В некотором
смысле, через базисы Грёбнера можно решать системы алгебраических уравнений
любой степени (на компьютере разумеется). В 20 веке была решенв проблема,
которой 35 веков, любая система алгебраических уравнений может быть решена.
Но изменился и сам предмет. На какое-то время алгебра стала пониматься как
изучение множеств с операциями. Этим занимались немецкие математики Дедекинд
– кольцами, Форбениус, Жарданова форма и т.д. Потом происходило ещё несколько
революций. 1920 Гильберт и Нётер. 1940-50 теория категорий и гомологическая
Алгебра. 1960 современная алгебраическая геометрия. Собственно книги по темам:
\begin{itemize}
    \item Шавалье "Введение в алгебру"
    \item Вандер Варден (1 перевод лучше второго)
\end{itemize}
Шавалье говорил "Алгебра играет такую же роль по отношению к математике,
какую математика играет по отношению к физике." Алгебра – язык, инструмент,
специфика которого в том, как что изучается. Сами математики разделяются по
чувству комфорта:
\begin{itemize}
    \item работают с числами – аналитики
    \item с картинами – геометры
    \item со словами – алгебраисты
\end{itemize}
\section{План}
Первый год алгебра будет играть служебную роль для остальных дисциплин. Так
как "всё" выражается через линейную алгебру. Те же дифференциальные уранения
могут быть хорошо аппроксимированы 1000000 линейных уравнений, как показал
Фадеев в 60х годах, этот метод используется и по сегодняшний день.
\subsection{I-ый семестр}
\begin{enumerate}
    \item Кольца и Арифметика
    \item Многочлены и Поля
    \item Модули и Векторные пространства (начала линейной алгебры)
    \item Группы
    \item Определители
\end{enumerate}
\subsection{II-ой семестр}
\begin{enumerate}
    \item Линейные операторы
    \item Квадратичные и Эрмитовы формы
    \item Кваторнионы
    \item Теория Групп
    \item Представления конечных групп
\end{enumerate}
\# многие вещи стали прикладной математикой
\subsection{III-ий семестр}
\begin{enumerate}
    \item Полилинейная алгебра
    \item Теория категорий
    \item Гомологическая алгебра
\end{enumerate}
\subsection{IV-ый семестр}
??

\chapter{Кольца и Арифметика Коммутативных Колец}
\section{Алгебраические операции}
Пусть $X\neq\varnothing$ – множество.

\textbf{Опр:} \emph{Бинарная алгебраическая операция} на $X$ – это отбражение
$X\times X\longrightarrow X$.

Мы также можем рассматривать:
\begin{itemize}
    \item $X\times Y\longrightarrow Z$ внешние операции
    \item $X\times X\times X\longrightarrow X$ тринарные
    \item ${*}\longrightarrow X$ нулярные
    \item $X\longrightarrow X$ унарные
\end{itemize}

И так, пусть у нас задано некое отображение $f: X\times X\longrightarrow X$.
Тогда функциональная запись может быть префиксной $f(x,y)$ или $fxy$, инфиксной
$xfy$, постфиксной $(x,y)f$ или $xyf$ или интерфиксная $<x,y>$ ещё много как.
Собственно в основном мы будем пользоваться инфиксной записью либо аддитивной
со значком $+$ или мультипликативной со значками $\times$ или $\cdot$ или $*$.

\textbf{Опр:} \emph{Нейтральным элементом} относительное некой операции $*$
называется элемент $e$, что $\forall x\in X, e*x = x = x*e$. Первая запись
означает, что элемент левый нейтральный, а вторая, что элемент – правый
нейтральный. В аддитивной записи нейтральный элемент обозначается $0_X$, а в
мультипликативной $1_x$.

\textbf{Опр:} Элемент $x'$ назывется симметричным к $x$, если $x'*x=e=x*x'$.
В аддитивной записи симметричный обычно называется противоположным, а в
мультипликативной – обратным.

\textbf{Опр:} Операция $*$ называется ассоциативной, если $\forall x, y, z \in
X (x*y)*z = x*(y*z)$. Раньше этот закон называли сочитательным. На самом деле
он обозначает функциональное уравнение $f(f(x,y),z) = f(x,f(y,z))$.

\textbf{Опр:} Операция $*$ называется коммутативной, если $x*y=y*x$, устаревшее
название – "переместительный закон".

На самом деле большая часть математики не коммутативна и мы будем отказываться
от неё. Многие обычные формулы за некоторым исключением существуют в
некоммутативном варианте и заучивать и использовать лучше сразу его. Например
$(1/f)' = -f^{-1}f'f^{-1}$ [проверьте]. 

Aбинкар писла, что есть 3 алгебры:
\begin{enumerate}
    \item Школьная – уравнение и многочлены
    \item Коледжная – группы, кольца, векторные пространства
    \item Университется – категории, функторы, комплексы, гомологии и когомологии
\end{enumerate}
Мы же постараемся дойти до университетского уровня.

Давайте посмотрим чем же важны ассоциантивные операции???. $e = e*e' = e'
\Rightarrow e=e'$ единственность единицы? В общем случае ассоциативносить
есть не везде, бесконечные матрицы умножаются не ассоциативно. В анализе на
такие бесконечные объекты накладывается условие сходимости, что делает операции
вновь ассоциативными.

Ассоциативность ещё хороша тем, что правые и левые обратные совпадают, то есть
$x*y=e$ и $x*z=e$, тогда $y=y*(x*z)=(y*x)*z=z$.

Про растановку скобок есть замечательный пример. Количество способов посчитать
неассоциативую операцию из $n$ множителей называется $n$-ым числом Каталана.
Предлагаю вам найти формулу для таких чисел или можете прочитать про них в
книге "Не совсем наивная теория множеств". C числами Каталана также связаны
числа Родригеса.

\end{document}
