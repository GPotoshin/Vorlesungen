\chapter{Кольца и Арифметика Коммутативных Колец}
\section{Алгебраические операции}
Пусть $X\neq\varnothing$ – множество.

\textbf{Опр:} \emph{Бинарная алгебраическая операция} на $X$ – это отображение
$X\times X\longrightarrow X$.

Мы также можем рассматривать:
\begin{itemize}
    \item $X\times Y\longrightarrow Z$ внешние операции
    \item $X\times X\times X\longrightarrow X$ тринарные
    \item ${*}\longrightarrow X$ нулярные
    \item $X\longrightarrow X$ унарные
\end{itemize}

И так, пусть у нас задано некое отображение $f: X\times X\longrightarrow X$.
Тогда функциональная запись может быть префиксной $f(x,y)$ или $fxy$, инфиксной
$xfy$, постфиксной $(x,y)f$ или $xyf$ или интерфиксная $<x,y>$ ещё много как.
Собственно в основном мы будем пользоваться инфиксной записью либо аддитивной
со значком $+$ или мультипликативной со значками $\times$ или $\cdot$ или $*$.

\textbf{Опр:} \emph{Нейтральным элементом} относительное некой операции $*$
называется элемент $e$, что $\forall x\in X, e*x = x = x*e$. Первая запись
означает, что элемент левый нейтральный, а вторая, что элемент – правый
нейтральный. В аддитивной записи нейтральный элемент обозначается $0_X$, а в
мультипликативной $1_x$.

\textbf{Опр:} Элемент $x'$ называется симметричным к $x$, если $x'*x=e=x*x'$.
В аддитивной записи симметричный обычно называется противоположным, а в
мультипликативной – обратным.

Заметим, что часта обратный элемент может быть обратим только с одной стороны.
Элемент $x$ называется обратимым слева (справа), если существует элемент $y$,
называемый левым (правым) обратным, что выполняется $y*x=e$ ($x*y=e$).

\textbf{Опр:} Операция $*$ называется ассоциативной, если $\forall x, y, z \in
X (x*y)*z = x*(y*z)$. Раньше этот закон называли сочетательным. На самом деле
он обозначает функциональное уравнение $f(f(x,y),z) = f(x,f(y,z))$.

\textbf{Опр:} Операция $*$ называется коммутативной, если $x*y=y*x$, устаревшее
название – "переместительный закон".

На самом деле большая часть математики не коммутативна и мы будем отказываться
от неё. Многие обычные формулы за некоторым исключением существуют в
некоммутативном варианте и заучивать и использовать лучше сразу его. Например
$(1/f)' = -f^{-1}f'f^{-1}$ [проверьте]. 

Aбинкар писал, что есть 3 алгебры:
\begin{enumerate}
    \item Школьная – уравнение и многочлены
    \item Коледжная – группы, кольца, векторные пространства
    \item Университетская – категории, функторы, комплексы, гомологии и когомологии
\end{enumerate}
Мы же постараемся дойти до университетского уровня.

Давайте посмотрим чем же важны ассоциативные операции???. $e = e*e' = e'
\Rightarrow e=e'$ единственность единицы? В общем случае ассоциативность
есть не везде, бесконечные матрицы умножаются не ассоциативно. В анализе на
такие бесконечные объекты накладывается условие сходимости, что делает операции
вновь ассоциативными.

Ассоциативность ещё хороша тем, что правые и левые обратные совпадают, то есть
$x*y=e$ и $x*z=e$, тогда $y=y*(x*z)=(y*x)*z=z$.

Про расстановку скобок есть замечательный пример. Количество способов посчитать
неассоциативую операцию из $n$ множителей называется $n$-ым числом Каталана.
Предлагаю вам найти формулу для таких чисел или можете прочитать про них в
книге "Не совсем наивная теория множеств". C числами Каталана также связаны
числа Родригеса.

На самом деле в школе уже были неассоциативные операции, например деление и
вычитание, но обычно их не используют как основные операции, хотя существуют
логические извращения, когда для групп задают только операцию разности, но мы
таким не будем заниматься. Другое не ассоциативной операцией было возведение в
степень. Причем нужно отметить, что устоявшаяся запись левонормирована (скобки
расставляют с начала с лева), а не более общепринятая правонормированная.

Другой знакомой вам не ассоциативной операцией является векторное произведение
(cross product). Но она удовлетворяет иному тождеству, которое часто заменяет
ассоциативность. \emph{Тождество Якоби} $(u\times v)\times w + (v\times w)
\times u+(w\times u)\times v = 0$. Об этом тождестве Арнольд говорил следующее
"Это тождество означает, что высоты треугольника пересекаются в одной точке."

\section{Моноиды}
\textbf{Опр:} $(X,*,e)$ называется \emph{мноидом}, если:
\begin{enumerate}
    \item $*$ accоциативна
    \item существует нейтральный элемент $e$
\end{enumerate}

\textbf{Опр:} \emph{Полугруппа} – это моноид без единицы.

\textbf{Примеры:}
$(\mathbb{N},\times,1)$, $(\mathbb{N}_0,+,0)$, $(\mathbb{N},\vee,1)$,
$(\mathbb{N}_0, \wedge, 0)$, $(2^Y,\cap,Y)$, $(2^Y,\cup,\varnothing)$,
$(2^Y,\triangle,\varnothing)$ – это всё моноиды.

Так как формально всегда можно присоединить единицу, то с алгебраической точки
зрения нет особой разницы изучать структуры без неё или с.

\subsection{Сокращения}
\textbf{Опр:} Элемент $x\in X$ называется \textbf{регулярным слева (справа)},
если на него можно сокращать слева (справа).
\[\forall x,z\in X,\, x*y=x*z\Rightarrow y=z\;(y*x=z*x\Rightarrow y=z)\]

\textbf{Лемма:} Элемент $x\in X$ обратимый слева/справа, регулярен слева/справа.

Прошу заметить, что в доказательстве СУЩЕСТВЕННУЮ роль играет ассоциативность.
В общем неассоциатвном случае это вообще говоря не верно. В продолжении мы
докажем теорему Фробениуса, которая покажет где нам нужно будет остановится в
построении расширений действительных чисел. Всем известны 4 тела, последнее из
которых неассоциативно $\mathbb{R}$, $\mathbb{C}$, $\mathbb{H}$, $\mathbb{O}$.
Так в $\mathbb{O}$, несмотря на отсутствие ассоциативности, любые два элемента
пораждают ассоциативную алгебру, то есть они альтернативны. Неассоциативность
всё же ограничена. Так что обычно этот ряд не продолжают. Теорема говорит, что
кроме первых трёх, нет других ассоциативных тел, а её обобщенная версия
показывает, что други конечномерных тел над $\mathbb{R}$ кроме этих 4х нет. Mы
получим последовательность $1,2,4,8$. На самом деле для 16 есть сединионы
Диксона, но хоть каждый элемент там обратим, но там также есть делители нуля.
Это показывает, что в неассоциативном случае наша интуиция перестаёт действовать.
На самом деле все наши обычные рассуждения основаны на ассоциативности. А если
её нет, то всему нужно учится с нуля.

\textbf{Докозательство:} Пусть $u*x=e$ и $x*y=x*z$, тогда $y=e*y=(u*x)*y=u*(x*y)=
u*(x*z)=(u*x)*z=e*z=z$. С другой стороны аналогично.

Но регулярный не обязательно обратим. Важнейшим классом моноидов являются те,
в которых каждый элемент обратим. Такие моноиды называются группами. [Теорема
Руфини-Абеля]. Моноиды, в которых операция коммутативна, называются
коммутативными, но группы называются абелевыми. Есть и Кайновы группы, кстати.

\section{Группы}
Группы – один из важнейших и первейший исторический пример алгебраических
систем. Формально группы были определены впервые Галуа, причем было видно как
это понятие у него в работах возникает. В начале он просто говорил "Grouper
les permutation", то есть группировать перестановки. Потом возник термин группа
перестановок, а в конце 19 века уже была развита теория групп, в первую очередь
конечных групп, но на самом деле не только. Так вот это одно из важнейших
математических понятий, структур. Придумать аксиом можно сколько угодно,
важность структур, которые реально изучают,  определяется не их сложностью и
многообразием. Важность и определение структуры для работающего математика –
это не набор свойств или аксиом, а набор содержательных не тривиальных примеров.
И такое замечательный алгебраист Адреан Альберт, русский кстати, но он работал
в Чикаго, хотя при этом был русским, так вот Адреан Алберт говорил, что
математическую структуру имеет смысл изучать, если есть 3 разных содеражательных
пример. Но вот у групп больше совершенно чем 3 разных содержательных примеров.

\textbf{Опр:} $(G,\cdot(=\text{mult}),.^{-1}(=\text{inv}),1)$ – группа, где\\
$\quad i)$ $\text{mult}: G\times G \longrightarrow G,\;(x,y)\mapsto x\cdot y$\\
$\quad ii)$ $\text{inv}: G\longrightarrow G,\;x\mapsto x^{-1}$\\
$\quad iii)$ $1\in G$\\
для которых выполнено:\\
$\quad$1) ассоциатиность $(x\cdot y)\cdot z=x\cdot (y\cdot z)$\\
$\quad$2) нейтральный элемент $1\cdot x = x = x\cdot 1$.\\
$\quad$3) обратный $\forall x\exists x',\;x\cdot x' = x'\cdot x = 1$

Многие буквы на самом деле говорящие. $G$ – group, $R$ – ring, $K$ – Körper,
$F$ - field, $A$ - anneau, $M$ - module, $V$ - vector space.

Переход от элемента к обратному является антиавтоморфизмом группы порядка 2.
$(xy)^{-1} = x^{-1}y^{-1}$ и $(x^{-1})^{-1}$. Очень важно помнить правильные
формулы. Мы надеваем педжак, а потом пальто, снимают их обычно в ином порядке.
Если $xy=yx$, то мы говорим, что они коммутируют.

Мы не будем заниматься ослаблением свойств групп, так как это абсолютно
бессмысленная деятельность.

\subsection{Элементарный свойства групп}
\begin{enumerate}
    \item Сокращение $xy = xz\Rightarrow y=z$ и $yx=zx\Rightarrow y=z$.
    \item Деление $\forall h,g\exists !x\;\text{т.ч.} hx=g$, (а именно
        $x=h^{-1}g$) $\exists!y$ т.ч. $xh=g$ (–//– $x=gh^{-1}$). Деление
        бывает справа и слева и это не одно и тоже.
\end{enumerate}

На самом деле есть всякие интересные обобщения группы. Если вы присмотретесь,
то нигде выше единица не фигурирует, в возможностях сокращения и деления. Так
вот можно изучать, и собственно изучались и очень важны структуры в которых
операция не ассоциативна, но тем не менее сокращение всегда возможно и деление
всегда возможно, это так называемые квази группы. Тогда огромная часть того,
что доказывается для групп имеет смысл и для квазигрупп. Например латинские
квадраты и есть квазигруппы. 

[ $\mathbb{Z}/(10)$ при обычном сложении мы отдельно смотрим на единицы и
отдельно на десятки и гомологии помогают из двух групп по модулю 10 получить
по модулю 100.]

Так что то, чем мы занимаемся, так это напоминание в Платоновском смысле. До
рождения человек знает всё, но вовремя забывает. И всё что он делает в жизни,
так это не учит, а вспоминает.

Примеры:
\begin{itemize}
    \item $(\mathbb{Z},+)$ бесконечная циклическая группа
    \item повороты n угольника $C_n\cong \mathbb{Z}/(n)\cong \mu_n$ конечная
        циклическая группа, классы вычитов по n и группа n-ых корней из 1 в $\mathbb c$
        $|C_n|=n$ – порядок группы.
    \item $D_n$ – диэдральная группа, группа симметрий правильного n-угольника.
        $|D_n|=2n$
    \item $S_n$ – симметрическая группа степени n. Основное, что нужно знать
        из теории множеств, так это то, что композиция ассоциативна. $\text{Bij}(X,X)$ 
        – cимметрическая группа биекций $X$. Так как мы обычно пишем функции
        слева, то композиция направлена в обратную сторону! Редко кто пишет в
        категорном стиле $(x)sin$. $(\text{Bij}(X), \circ, .^{-1}, id_X)$. Если
        f – биективно $\iff$ f – обратимо. [Вы должны читать не то, что здесь
        написанно, а то, что я думаю.] Пусть $\underline{n}=\{1,2,...,n\}$. Так
        вот $S_n=S_{\underline{n}}$. $|S_n|=n!$. Интересная книжка "Три
        поросёнка" Мацумаса, где обсуждается сколько способов рассадить 3
        парасёнка по 3–м домикам. Разные случаи там могут быть, поросята могут
        сидеть вместе, в каком порядке их съедает волк итд. Это всё обсуждают
        волк и его друг жаб Сократ. Когда мы увидим действие групп, то поймём,
        что любая конечная группа вкладывается в некотрый $S_n$.
    \item $(\mathbb{R}, +)$, $(\mathbb{Q}, +)$, ...
    \item $(\mathbb{R}^*, \cdot)$, $(\mathbb{R}_{>0},\cdot)$
    \item $\pi$ группа углов. Её можно получить на окружности, зафиксировав 1
        и проводя параллельные прямые. Возьмём две точки на окружности, проведем
        через них прямую, проведём параллельную к ней через 1. Эта прямая
        пересечет окружность в новой точке, назавём её произведение двух
        предыдущих. Эта операция образует топологическую группу изоморфную 
        $\mathbb{R}/\mathbb{Z}$.
\end{itemize}

\textbf{Опр:} G – абелева, если умножение коммутативно.

\textbf{Опр:} $\varphi: H\longrightarrow G$ – гомоморфизм групп, если
$\varphi(xy)=\varphi(x)\varphi(y)$.

Пример: $\exp: (\mathbb{R}, +)\longrightarrow (\mathbb{R}_{>0},\cdot)$ -
гомоморфизм и более того изоморфизм. Аналогично про логарифм.

На самом деле разные классы морфизмов имеют разные названия. Инъективный
гомоморфизм групп называется мономорфизмом, но в общем случае это совершенно
не тоже самое, что инъективный гомоморфизм. Это не так например для колец.
Cюрьективный называвает эпиморфизмом, биективный называется изоморфизмом. В
предыдущем примере мы видели именно изоморфизм. В топологии это совершенно не
так, поэтому она существенно сложнее. Обратное тоже должно быть морфизмом! Если
между двумя объектами есть изморфизм, то говорят, что они изоморфны. В этом
случае объекты не различаются. Гомоморфизм на себя называется эндоморфизмом.
Изоморфизм – автоморфизмом. Приставка анти- означает замену порядка.

\section{Кольца. Первые примеры}

Кольца тоже одно из основных понятий алгебры и математики в целом, так как
очень много различных объектов, вознакающих в разных ситуациях, имеют струтуру
кольца.

\textbf{Опр:} Кольцо – это множество с двумя операциями сложением и умножением.
Относительно сложения кольцо образует абелеву группу, а умножение дистрибутивно
(с двух сторон) относительно сложения.

$.^+$ – функтор из категории колец в категорию абелевых групп. То есть $R^+$
это тоже самое множество, только без умножения. Функторы, грубо говоря, это
срелки которы все объекты одной категории превращают в объекты другой. А
морфизмы одной категории согласовано превращаются в морфизмы другой.

Обычно рассматриваются ассоциативные кольца с единицей. То есть умножение
ассоциативно, и есть нейтральный элемент относительно умножения. Так как мы
хотим в будущем вкладывать область целостности в поле частных, то нам нужно,
чтобы нуль и один не совпадали. Иначе вложение не удастся. В этой главе
большинство колец ещё будут коммутативными (умножение).

Бывают ли коммутативные не ассоциативные кольца? Да, например Грайс использовал
такое при построении большого монстра.

\textbf{Опр:} Ассоциативное кольцо с 1 называется телом (fr. corp, en. skew-field),
если каждый элемент, кроме нуля, обратим. То есть моноид получаемый из кольца
выкидыванием нуля и сохранением только умножения $R^\bullet = R\backslash\{0\}$ является
группой. Также будет ещё один функтор $R^*$ – группа обратимых элементов.

\textbf{Опр:} $R$ называется полем, если оно коммутативное тело.

Примеры:
\begin{itemize}
    \item $\mathbb{Z}$ – целые числа (Zahlen)
    \item $\mathbb{Z}[1/2]$ – кольцо двоичных дробей
    \item $\mathbb{Z}[1/2, 1/3, 1/5]\subseteq\mathbb{Q}$ (quotient)
    \item $\mathbb{Q}$, $\mathbb{R}$, $\mathbb{C}$ – поля
    \item $\mathbb{Z}/n\mathbb{Z}$ – кольцо классов вычитов по модулю n
\end{itemize}

Мы писали фактор кольцо по идеалу $I\unlhd R$. Математике вам нужно учиться,
как маленькие дети учатся языку. Маленьким детям не объясняют каждое слово, им
говорят "Вот посмотри, это фактор кольцо, вот это к-цикл". И когда вы услышите
5 раз использование этих слов, то вы будете понимать, что они значат, даже если
вы не будете знать определения. Определение можно будет выучить уже позже,
потому что за всем этим должно стоять понимание, если его не будет, то все
слова и формулы – это просто дым. Математика живет только в понимании, сами
последовательности буковок ничего не значат.

\subsection{Таблицы Кэли Cayley ($\neq$ Келли Kelley)}
Артур Кэли провел детство в Питербурге до 12 лет, (но Кантор там родился). Кэли
разумеется Британский математик, и по началу он был юристом, поверенным в
наследственных делах, но за это время написал порядком 200 работ и конечно
потом стал профессором в Кембридже. Он был одним из классиков алгебры и многие
вещи из линейной алгебры, теории групп восходят к нему.

Как заметил Галуа, $\mathbb{Z}/4\mathbb{Z}\ncong\mathbb{F}_4$. Поэтому посмотрим
на $\mathbb{Z}/4\mathbb{Z}=\{[0], [1], [2], [3]\}=\{0,1,2,3\}$
\begin{multicols}{2}
    \hspace{3ex}
    \setlength\extrarowheight{3pt}
    \begin{tabular}{c | c c c c}
        + & 0 & 1 & 2 & 3 \\
        \cline{1-5}
        0 & 0 & 1 & 2 & 3 \\
        1 & 1 & 2 & 3 & 0 \\
        2 & 2 & 3 & 0 & 1 \\
        3 & 3 & 0 & 1 & 2 \\
    \end{tabular}
    \columnbreak
    \hspace{8ex}
    \begin{tabular}{c | c c c c}
        \times & 0 & 1 & 2 & 3 \\
        \cline{1-5}
        0 & 0 & 0 & 0 & 0 \\
        1 & 0 & 1 & 2 & 3 \\
        2 & 0 & 2 & 0 & 2 \\
        3 & 0 & 3 & 2 & 1 \\
    \end{tabular}
\end{multicols}

Мы выполяем операции над представителями классов, а затем берём остаток и
записываем класс через его представителя. Можно заметить, что первая таблица
является латинским квадратом! То есть в каждой строке и столбце элементы
различны. Это ровно означает, что эта операция задаёт квазигруппу, то есть, что
можно сокращать. В группе мы можем сокращать, а значит всё верно и у нас должен
был получиться латинский квадрат.

Простейший следствия аксиом кольца $0*x = 0 = x*0$. В элементарных учебниках
любят такие теоремы, я не буду говорить где такое есть, но если вы поищете, то
найдете такого много.

Обратите внимение! $2\neq0$, но $2*2=0$. Ненулевые числа, что не являются
нулями, но в произведении дают нуль называются \emph{делителями нуля}, а те,
что в некоторой целой степени дают нуль, \emph{нильпотентами}. Нильпотенты, это
то, что в анализе называется бесконечно малым некоторого порядка, их вперые
начали рассматривать в 17 веке де Ферма и фон Лейбниц. [проверьте или проверьте]
Они начали рассматривать буковку $d$, что в некоторой степени равна нулю. А
потом следующий поколения забыли, что эта буковка значит и стали писать $d$ от
$x$ и начали говорить, что это какой-то дифференциал и так далее.

Если из таблицы умножения мы выкинем нулевые строки, то не получим латинского
квадрата, а значит умножение не образует группу и $\mathbb{Z}/4\mathbb{Z}$ не
поле.

На самом деле на любой абелевой группе можно ввести структуру кольца
ассоциативного и коммутативного, но при этом без единицы. Условие существования
единицы – очень сильное условие, хотя к любому кольцу можно добавить 1.

\subsection{Кольцо с нулевым умножением}
Если $A$ это у нас любая абелева группа по сложению, тогда определим умножение
следующим образом $x\cdot y\mapsto 0$. Вопрос какие операции можно ввести –
отдельный.

\subsection{Булево кольцо множеств}
Пусть $X$ – любое множество, тогда можем рассмотреть $R=2^X$ множество всех
подмножеств $X$, тогда на $R$ можно ввести следующие операции сложения и
умножения. Для $Y,Z\subseteq X$, то положим $Y+Z=Y\triangle Z$, что называется
или Булевой суммой или симметрической разностью. Тогда нетрудно осознать, что
такая сумма коммутативна, ассоциативна, $0=\varnothing$. В нём $2y=0$, то есть
характеристика равна 2. В качестве произведения возьмём пересечение, тогда в
этом кольце $1=X$. Такое умножение, кстати, коммутативно, ассоциативно,
дистрибутивно относительно сложения. Заметим, что в этом кольце $y^2=y$, то
есть любой элемент является \emph{идемпотентом}. Eсть целый раздел, теория
решеток, где такие кольца играют большую роль.

\subsection{Следсвия и аксиом}
Так как сложение образует группу, то в кольце определено вычитание. Тогда все
обычные свойства будут выполнятся.
\begin{center}
    $x(y-z) = xy-xz$\\
    $(x-y)z = xz-yz$\\
    $(-1)x = -x$\\
    $(-x)(-y) = xy$\\
    ...
\end{center}
Моральный смысл такое, что коммутативное ассоциативное кольцо с единицей – это
то место, где выполняются все правила школьной алгебры, кроме деления и
сокращения. С делением всё сложнее, так как в отсутствие коммутативности деление
слева и справа это совсем не одно и тоже $x/y=xy^{-1}\neq y\backslash x=y^{-1}x$.

Давайте на кое-что посмотрим что можно в кольце $(x+y)(x-y)=x^2+yx-xy-y^2$.
Назавём аддитивным коммутатором следующее $[y,x]=yx-xy$. Если элементы
коммутируют, то получится школьная формула, но вообще это может быть неверным.
Точно также в коммутативном кольце выполняется формула бинома Ньютона, но не
верна в некоммутативном! $(x+y)^2=x^2+xy+yx+y^2$. В частности подставлять
матрицы в школьные формулы НЕЛЬЗЯ! Так многие формулы во многих курсах даются
неверными. $(fg)'=f'g+fg'$ обратите внимание на верный порядок! Если операция
не коммутативна, то нужно очень аккуратно следить за порядком сомножителей. И
тогда гораздо лучше учить сразу правильные некоммутативные формулы.

\section{Простейшие конструкции колец}
\begin{itemize}
    \item Из коммутативного ассоциативного кольца с 1 $R$ можно построить
        кольцо многочленов от одной переменной $R[t]$.
\end{itemize}

