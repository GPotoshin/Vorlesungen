\chapter{Кольца и Арифметика Коммутативных Колец}
\section{Алгебраические операции}
Пусть $X\neq\varnothing$ – множество.

\textbf{Опр:} \emph{Бинарная алгебраическая операция} на $X$ – это отбражение
$X\times X\longrightarrow X$.

Мы также можем рассматривать:
\begin{itemize}
    \item $X\times Y\longrightarrow Z$ внешние операции
    \item $X\times X\times X\longrightarrow X$ тринарные
    \item ${*}\longrightarrow X$ нулярные
    \item $X\longrightarrow X$ унарные
\end{itemize}

И так, пусть у нас задано некое отображение $f: X\times X\longrightarrow X$.
Тогда функциональная запись может быть префиксной $f(x,y)$ или $fxy$, инфиксной
$xfy$, постфиксной $(x,y)f$ или $xyf$ или интерфиксная $<x,y>$ ещё много как.
Собственно в основном мы будем пользоваться инфиксной записью либо аддитивной
со значком $+$ или мультипликативной со значками $\times$ или $\cdot$ или $*$.

\textbf{Опр:} \emph{Нейтральным элементом} относительное некой операции $*$
называется элемент $e$, что $\forall x\in X, e*x = x = x*e$. Первая запись
означает, что элемент левый нейтральный, а вторая, что элемент – правый
нейтральный. В аддитивной записи нейтральный элемент обозначается $0_X$, а в
мультипликативной $1_x$.

\textbf{Опр:} Элемент $x'$ назывется симметричным к $x$, если $x'*x=e=x*x'$.
В аддитивной записи симметричный обычно называется противоположным, а в
мультипликативной – обратным.

\textbf{Опр:} Операция $*$ называется ассоциативной, если $\forall x, y, z \in
X (x*y)*z = x*(y*z)$. Раньше этот закон называли сочитательным. На самом деле
он обозначает функциональное уравнение $f(f(x,y),z) = f(x,f(y,z))$.

\textbf{Опр:} Операция $*$ называется коммутативной, если $x*y=y*x$, устаревшее
название – "переместительный закон".

На самом деле большая часть математики не коммутативна и мы будем отказываться
от неё. Многие обычные формулы за некоторым исключением существуют в
некоммутативном варианте и заучивать и использовать лучше сразу его. Например
$(1/f)' = -f^{-1}f'f^{-1}$ [проверьте]. 

Aбинкар писла, что есть 3 алгебры:
\begin{enumerate}
    \item Школьная – уравнение и многочлены
    \item Коледжная – группы, кольца, векторные пространства
    \item Университется – категории, функторы, комплексы, гомологии и когомологии
\end{enumerate}
Мы же постараемся дойти до университетского уровня.

Давайте посмотрим чем же важны ассоциантивные операции???. $e = e*e' = e'
\Rightarrow e=e'$ единственность единицы? В общем случае ассоциативносить
есть не везде, бесконечные матрицы умножаются не ассоциативно. В анализе на
такие бесконечные объекты накладывается условие сходимости, что делает операции
вновь ассоциативными.

Ассоциативность ещё хороша тем, что правые и левые обратные совпадают, то есть
$x*y=e$ и $x*z=e$, тогда $y=y*(x*z)=(y*x)*z=z$.

Про растановку скобок есть замечательный пример. Количество способов посчитать
неассоциативую операцию из $n$ множителей называется $n$-ым числом Каталана.
Предлагаю вам найти формулу для таких чисел или можете прочитать про них в
книге "Не совсем наивная теория множеств". C числами Каталана также связаны
числа Родригеса.
